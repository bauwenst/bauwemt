% !TeX spellcheck = en_GB
\chapter{Introduction}\label{chap:intro}
The point of \repo is to give you a bare-bones example of what a thesis could look like. It pretends to be a short thesis itself about how to use \repo and what goes where in a thesis. You can read it as straight source code, or compile it and read it as a PDF showing what your thesis could look like in the end.

\section{What do you put in an introductory chapter?}
The introductory chapter sets the tone for the rest of the thesis, both in terms of writing style and in terms of topic. You give the uninitiated reader a first look into the world you will be studying, assuming they only know as much as someone who has passed the common core of the CS master. You establish what that world is, and lead the reader to realise that this world you describe has a need for your thesis.

What prompted me to make \repo? The fact that \texttt{kulemt} really just kinda sucks. (Don't worry if you don't know \texttt{kulemt}, it will be covered in \autoref{chap:literature}.)

\section{Rest of the document}
It is customary to end your introductory chapter with a birds-eye overview of the thesis. The difference with your abstract is that here you \emph{should} refer to specific parts/chapters, whereas you \emph{should not} discuss the conclusions you draw at the end of them.
\begin{quote}
	\fullref{pt:background} contains this chapter, a literature study about alternatives available to engineering students to lay out their thesis, and the question \repo answers.
\end{quote}
\begin{quote}
	\fullref{pt:contribution} comprises installation instructions and an overview of \repo, as well as an elaborate showcase of how to achieve certain outputs with it (for example, how to lay out algorithm pseudocode). The conclusion recaps the core ideas.
\end{quote}

In \fullref{pt:apx} you can give additional information that isn't pertinent to the main body (or just doesn't fit inside the page limit):
\begin{quote}
	\fullref{apx:neuralnetworks} gives examples of drawing neural networks.
\end{quote}
