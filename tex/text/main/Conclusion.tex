% !TeX spellcheck = nl_NL
\chapter{Summary \& Conclusion}
There is a difference between the summary and the conclusion.
\begin{itemize}
	\item The summary recapitulates what the thesis covered. For each chapter, you say what its purpose was, but \emph{also} give concrete take-aways the reader should remember from the different parts.
	\item The conclusion answers the research questions briefly, and then looks into the future of the field of study.
\end{itemize}

\section{Summary}
\autoref{chap:intro} explained how \repo is like a short thesis that can be read as source code or as a PDF. It also explained what a thesis introduction should accomplish.
\segsep
\autoref{chap:literature} gave a brief overview of alternative thesis templates. We saw that the most important existing template was kulemt and the reasons why I don't like it.
\segsep
\autoref{chap:questions} converged on the fundamental question, which is whether we can make an engineering master's thesis template better than kulemt.
\segsep 
\autoref{chap:bauwemt} walked through the installation of \repo. It then discussed how the project is organised into a preamble, a main text, and resources. Finally, it also gave practical advice on how to write the thesis, in particular stressing that you should always save except while compiling.
\segsep
\autoref{chap:functionality} lastly showed off all the different outputs that can be achieved out of the box with \repo. Apart from going over existing commands for equation alignment, hyperlinking, citations, figures, tables, algorithms, boxes, code, menu navigation, and line break control, we also covered new commands like \verb|blockquote| for nicer quotation, \verb|\pagelabel| for correctly linking back to the page of an object, and \verb|\emph*| and \verb|\Emph*| for adding terms to the index.

\section{Conclusion}
I asked in \autoref{chap:questions} if a better template could be crafted than kulemt. From all of the above, the answer is yes.

\section{Future work}
You can now provide the reader with a list of research questions you yourself would like to see answered, knowing all that you know now and having improved the state-of-the-art. Of course, \repo is perfect and hence there is no future work \co{;-)}.
