%%%%%%%%%%%%%%%%%%%%%%%
%%% CUSTOM COMMANDS %%%
%%%%%%%%%%%%%%%%%%%%%%%
\usepackage{suffix}

% \fullref command outputting a link with the autoref link AND the full name of whatever is linked; practically, this is most useful for referencing parts (e.g. `Part III: Statistics')
\newcommand*{\fullref}[1]{\hyperref[{#1}]{\autoref*{#1}: \nameref*{#1}}}

% By default, \autopageref links to the first anchor that appears above the \label it refers to, rather than to the exact location of the \label (despite outputting the correct page number!). So, say you have a section with no subsections that runs for many pages, \autopageref to a \label you place in the text will just link to the start of the section. \pagelabel anchors the label. https://tex.stackexchange.com/a/193392/203081
\newcommand{\pagelabel}[1]{\phantomsection\label{#1}}

% \emph*, which applies \emph and adds its content to the index.
% 	- Index functionality: https://tex.stackexchange.com/a/50773/203081
%	- Lowercasing based on: https://tex.stackexchange.com/a/380505/203081
\def\lowmystringaux#1#2\relax{\lowercase{#1}#2}
\newcommand\firstToLow[1]{\lowmystringaux#1\relax}

\WithSuffix\newcommand\emph*[1]{% Lowercases the first letter.
	\emph{#1}%
	\index{#1@\firstToLow{#1}}%  {what it's indexed as @ what it shows up as}. As it turns out, it is VERY difficult to insert the lowercased string into \index without the latter seeing the \ of the command that does the lowercasing.
}
\WithSuffix\newcommand\Emph*[1]{\emph{#1}\index{#1}}  % Keeps capitalisation.

% The current and next year
\def\thisyear{\the\numexpr\year+0\relax}
\def\nextyear{\the\numexpr\year+1\relax}

% Scaling in math mode (use in math mode, and specifying a scale factor i.o. a length)
\newcommand\scalemath[2]{\scalebox{#1}{\mbox{\ensuremath{\displaystyle #2}}}}

% Small separation line
\newcommand*\segsep{% Use most times
	\vspace{-0.6em}\\\rule{1cm}{0.4pt}\\
}

% Font size between \tiny and \scriptsize (https://tex.stackexchange.com/a/432684/203081)
\makeatletter
\newcommand\notsotiny{\@setfontsize\notsotiny\@vipt\@viipt}
\makeatother

% Math
\newcommand\Log[1]{\mathop{{}^{#1}\!\log}}  % Benelux logarithm
\renewcommand{\emptyset}{\varnothing}  % The better emptyset
\newcommand{\D}{\mathcal{D}}
\newcommand{\T}{\mathcal{T}}
\newcommand{\N}{\mathbb{N}}
\newcommand{\R}{\mathbb{R}}
\newcommand{\E}{\mathbb{E}}
\newcommand{\ft}{\mathfrak{t}}  % fraktur t
\newcommand{\elltwo}{\mathbb{L}^{\!2}}
\newcommand{\matr}[1]{\bm{#1}}
\newcommand{\tmatr}[1]{\bm{#1}^\intercal}
\newcommand{\listindex}[1]{\co{[}#1\co{]}}
\DeclareMathOperator{\softmax}{softmax}
\DeclareMathOperator{\Li}{\mathcal{L}}
\DeclareMathOperator{\uni}{\cup}
\newcommand{\ind}[1]{{\mathds{1}}\{#1\}}  % Curly braces because square brackets without any prefix already mean "convert bool to int", so a prefix would become pleaonastic. See https://en.wikipedia.org/wiki/Iverson_bracket
\renewcommand{\Re}{\text{Re}}  % Ain't nobody got time for RealPart(). I use it for recall.

% Easy superscript for centuries, e.g. 20\up{th}
\let\up\textsuperscript
\let\down\textsubscript

\usepackage{relsize}
\renewcommand{\_}{\textscale{.5}{\textunderscore}}  % Shorter underscore https://tex.stackexchange.com/a/128817/203081. DO NOT USE THIS IN MATH MODE, IT WILL CRASH THE COMPILER AND TEXSTUDIO.
\newcommand{\specialtoken}[1]{\langle \texttt{#1}\rangle}
\newcommand{\unk}{\specialtoken{UNK}}
\newcommand{\eos}{\specialtoken{EOS}}
\newcommand{\sow}{\textsc{sow}\xspace}
\newcommand{\eow}{\textsc{eow}\xspace}
\newcommand{\ellipsis}{\textsc{[...]}\xspace}

% Text styles
\newcommand{\ex}[1]{\textsl{#1}}   % Example
\newcommand{\co}[1]{\texttt{#1}}   % Code
\newcommand{\lxm}[1]{\textsc{#1}}  % Lexeme
\newcommand{\tk}[1]{\textsf{#1}}   % Tokeniser
\newcommand{\textgreek}[1]{{\fontspec{Linux Libertine O}#1}}
\newcommand{\textrussian}[1]{{\fontspec{Linux Libertine O}#1}}

% Hebrew support (any/all of these statements are necessary; took me a long time to throw this together)
\babelfont[hebrew]{rm}[Path=./pre/fonts/,Extension=.ttf]{SILEOTSR}  % Note: this *replaces* calls to \fontspec and \newfontfamily, hence why its last two arguments are identical (https://tex.stackexchange.com/a/665859/203081). Declaring the font with \fontspec and then leaving out the path in babelfont DOES NOT WORK. Aside: it's not a problem if you already have this font installed (from https://software.sil.org/ezra/ or on Overleaf).
\babelprovide[onchar=fonts ids]{hebrew}  % This should allow writing Hebrew Unicode without wrapping it in \texthebrew
\DeclareLanguageMapping{hebrew}{english}  % This means BibLaTeX doesn't complain about "language not supported".
\newcommand{\texthebrew}[1]{\foreignlanguage{hebrew}{#1}}

% Abbreviations
\newcommand{\roberta}{Ro\textsc{BERT}a\xspace}
\newcommand{\berttiny}{\textsc{bert}\textsubscript{\textsc{tiny}}\xspace}
\newcommand{\repo}{\textsc{bauwemt}\xspace}

% Wikipedia superscripts
\newcommand{\eng}{Eng.:\,}  % To be used explicitly without braces (\eng, not \eng{})
\newcommand{\ned}{Ned.:\,}
\newcommand{\cn}{\textsuperscript{\color{blue}[citation needed]}}

% Support Unicode characters ⸤ and ⸥ (https://tex.stackexchange.com/a/679336/203081)
\usepackage{newunicodechar}
\newfontface{\lowbrackets}{Noto Sans}
\NewDocumentCommand{\blhb}{}{{\lowbrackets\symbol{"2E24}}}
\NewDocumentCommand{\brhb}{}{{\lowbrackets\symbol{"2E25}}}
\newunicodechar{⸤}{\blhb}
\newunicodechar{⸥}{\brhb}

% Combine list of figures and tables. Generalised version of https://tex.stackexchange.com/a/14545/203081
\makeatletter
\providecommand\phantomsection{}% for hyperref
\newcommand{\listofillustrations}{%
	\chapter{\illustrationsname}%
	\phantomsection
	%\addcontentsline{toc}{chapter}{Lijst van illustraties}%
	\section*{\listfigurename}%
	\phantomsection
	\addcontentsline{toc}{section}{\listfigurename}%
	\@starttoc{lof}%
	\bigskip
	\section*{\listtablename}%
	\phantomsection
	\addcontentsline{toc}{section}{\listtablename}%
	\@starttoc{lot}%
	\bigskip
	%\section*{\listalgorithmname}%
	%\phantomsection
	\addcontentsline{toc}{section}{\listalgorithmname}%
	% So, for some reason, \@starttoc{loa} doesn't want to work (it outputs garbled titles). Luckily I found this fix https://tex.stackexchange.com/a/198860/203081 as an alternative. It lets \listofalgorithms do its thing as usual, except we ninja-edit \chapter* to be \section*.
	\begingroup
	\let\chapter\section% Temporary
	\listofalgorithms
	\endgroup
	%\@starttoc{loa}%
}
\makeatother

% Todonotes that doesn't kill you when you include it. TODO: Could be better, e.g. anti-overlap.
\let\marginnote\someundefinedcommand
\usepackage{marginnote}
%\reversemarginpar
\newcommand{\marginboxTB}[2]{\marginnote{%
	\begin{tikzpicture}%
		\node[draw,fill=#1,align=left,rounded corners,text width=0.5\marginparwidth] () {#2};%Need to align=... to be breakable. https://tex.stackexchange.com/a/31100/203081
	\end{tikzpicture}}[-1.75em]}% Move up by one line.

\newcommand{\todo}[1]{\marginboxTB{red!15}{#1}}

% Gradient-coloured table cells: surround cell content by \tgrad. Based on a mix of answers at https://tex.stackexchange.com/q/174998/203081.
% - These packages have already been loaded, but you'll need them if you want to use this in a standalone project:
%\usepackage[table]{xcolor}
%\usepackage{arydshln}
%\usepackage{hhline}  % There's a fatal flaw in arydshln's partial table lines, \cline and \cdashline, namely that they don't move cells down to make a gap into which to insert their line, unlike \hline. As a result, coloured cells cover those lines (see https://tex.stackexchange.com/a/603623/203081). A solution is using \hhline syntax (https://tex.stackexchange.com/a/121477/203081).
\usepackage{etoolbox}
\usepackage{pgf}
\usepackage{xargs}

% - These are for making minuses look like the horizontal bar of a plus (I made this myself!)
\usepackage{xstring}
\newcommand{\longerminus}{\scalebox{1.9}[1.0]{-}\kern-0.5pt}
\newcommand{\makeminuslonger}[1]{\StrSubstitute{#1}{-}{\longerminus}}

% - Definitions
\definecolor{high}{HTML}{03AC13}
\definecolor{mid}{HTML}{F7E379}
\definecolor{low}{HTML}{ec462e}
\newcommand*{\opacity}{80}

% - Cell command
\newcommandx{\tgrad}[4][1=0.0, 2=0.5, 3=1.0]{%
    \ifdim #4 pt > #2 pt%
        \pgfmathparse{max(min(100.0*(#4-#2)/(#3-#2),100.0),0)}%
        \xdef\PercentColor{\pgfmathresult}%
        \cellcolor{high!\PercentColor!mid!\opacity}\makeminuslonger{#4}%
    \else
        \pgfmathparse{max(min(100.0*(#2-#4)/(#2-#1),100.0),0)}%
        \xdef\PercentColor{\pgfmathresult}%
        \cellcolor{low!\PercentColor!mid!\opacity}\makeminuslonger{#4}%
    \fi
}

% - Optional: add a new tabular column type, "R", that applies tgrad to all cells automatically. Only works for fully numeric columns though; no column headers allowed.
\usepackage{collcell}
\newcolumntype{R}{>{\collectcell\tgrad}c<{\endcollectcell}}

% String truncation: keep the first m and last n characters of a given string. https://tex.stackexchange.com/a/726030/203081
\newcommand{\TruncateString}[3]{\StrLeft{#3}{\expandafter #1}\ldots\StrRight{#3}{#2}}
