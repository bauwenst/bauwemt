%%%%%%%%%%%%%%%%%%%%%%%
%%% CUSTOM COMMANDS %%%
%%%%%%%%%%%%%%%%%%%%%%%
\usepackage{suffix}

% \fullref command outputting a link with the autoref link AND the full name of whatever is linked; practically, this is most useful for referencing parts (e.g. `Part III: Statistics')
\newcommand*{\fullref}[1]{\hyperref[{#1}]{\autoref*{#1}: \nameref*{#1}}}

% \emph*, which applies \emph and adds its content to the index. https://tex.stackexchange.com/a/50773/203081
\WithSuffix\newcommand\emph*[1]{\emph{#1}\index{#1}}

% The current and next year
\def\thisyear{\the\numexpr\year+0\relax}
\def\nextyear{\the\numexpr\year+1\relax}

% Scaling in math mode (use in math mode, and specifying a scale factor i.o. a length)
\newcommand\scalemath[2]{\scalebox{#1}{\mbox{\ensuremath{\displaystyle #2}}}}

% Small separation line
\newcommand*\segsep{% Use most times
	\vspace{-0.6em}\\\rule{1cm}{0.4pt}\\
}

% Font size between \tiny and \scriptsize (https://tex.stackexchange.com/a/432684/203081)
\makeatletter
\newcommand\notsotiny{\@setfontsize\notsotiny\@vipt\@viipt}
\makeatother

% Math
\newcommand\Log[1]{\mathop{{}^{#1}\!\log}}  % Benelux logarithm
\renewcommand{\emptyset}{\varnothing}  % The better emptyset
\newcommand{\D}{\mathcal{D}}
\newcommand{\T}{\mathcal{T}}
\newcommand{\N}{\mathbb{N}}
\newcommand{\R}{\mathbb{R}}
\newcommand{\E}{\mathbb{E}}
\newcommand{\ft}{\mathfrak{t}}  % fraktur t
\newcommand{\elltwo}{\mathbb{L}^{\!2}}
\newcommand{\matr}[1]{\bm{#1}}
\newcommand{\tmatr}[1]{\bm{#1}^\intercal}
\newcommand{\listindex}[1]{\cd{[}#1\cd{]}}
\DeclareMathOperator{\softmax}{softmax}
\DeclareMathOperator{\Li}{\mathcal{L}}
\DeclareMathOperator{\uni}{\cup}
\newcommand{\ind}[1]{{\mathds{1}}\{#1\}}  % Curly braces because square brackets without any prefix already mean "convert bool to int", so a prefix would become pleaonastic. See https://en.wikipedia.org/wiki/Iverson_bracket
\renewcommand{\Re}{\text{Re}}  % Ain't nobody got time for RealPart(). I use it for recall.

% Easy superscript for centuries, e.g. 20\up{th}
\let\up\textsuperscript
\let\down\textsubscript

\usepackage{relsize}
\renewcommand{\_}{\textscale{.5}{\textunderscore}}  % Shorter underscore https://tex.stackexchange.com/a/128817/203081. DO NOT USE THIS IN MATH MODE, IT WILL CRASH THE COMPILER AND TEXSTUDIO.
\newcommand{\specialtoken}[1]{\langle \texttt{#1}\rangle}
\newcommand{\unk}{\specialtoken{UNK}}
\newcommand{\eos}{\specialtoken{EOS}}
\newcommand{\sow}{\textsc{sow}\xspace}
\newcommand{\eow}{\textsc{eow}\xspace}
\newcommand{\ellipsis}{\textsc{[...]}\xspace}

% Text styles
\newcommand{\ex}[1]{\textsl{#1}}   % Example
\newcommand{\cd}[1]{\texttt{#1}}   % Code
\newcommand{\lxm}[1]{\textsc{#1}}  % Lexeme
\newcommand{\tk}[1]{\textsf{#1}}   % Tokeniser
\newcommand{\greek}[1]{{\fontspec{Linux Libertine O}#1}}
\newcommand{\russian}[1]{{\fontspec{Linux Libertine O}#1}}
\newcommand{\chinese}[1]{{\fontspec{Linux Libertine O}#1}}

% Hebrew support (any/all of these statements are necessary; took me a long time to throw this together)
\babelfont[hebrew]{rm}{SILEOT.ttf}  % You need to install this. Overleaf has it automatically. https://software.sil.org/ezra/
\babelprovide[onchar=fonts ids]{hebrew}  % This should allow writing Hebrew Unicode without wrapping it in \texthebrew 
\DeclareLanguageMapping{hebrew}{english}  % This means BibLaTeX doesn't complain about "language not supported".
\newcommand{\texthebrew}[1]{\foreignlanguage{hebrew}{#1}}

% Abbreviations
\newcommand{\bert}{\textsc{bert}\xspace}
\newcommand{\roberta}{Ro\textsc{BERT}a\xspace}
\newcommand{\robbert}{Rob\textsc{BERT}\xspace}
\newcommand{\berttiny}{\textsc{bert}\textsubscript{\textsc{tiny}}\xspace}
\newcommand{\gptFour}{\textsc{gpt-4}\xspace}
\newcommand{\gptThree}{\textsc{gpt-3}\xspace}
\newcommand{\gptTwo}{\textsc{gpt-2}\xspace}
\newcommand{\gpt}{\textsc{gpt}\xspace}
\newcommand{\bleu}{\textsc{bleu}\xspace}
\newcommand{\bpe}{\textsc{bpe}\xspace}
\newcommand{\wpm}{\textsc{wpm}\xspace}
\newcommand{\bpeNL}{\textsc{bpe}\down{nl}\xspace}

% Wikipedia superscripts
\newcommand{\eng}{Eng.:\,}  % To be used explicitly without braces (\eng, not \eng{})
\newcommand{\ned}{Ned.:\,}
\newcommand{\cn}{\textsuperscript{\color{blue}[citation needed]}}

% Support Unicode characters ⸤ and ⸥ (https://tex.stackexchange.com/a/679336/203081)
\usepackage{newunicodechar}
\newfontface{\lowbrackets}{Noto Sans}
\NewDocumentCommand{\blhb}{}{{\lowbrackets\symbol{"2E24}}}
\NewDocumentCommand{\brhb}{}{{\lowbrackets\symbol{"2E25}}}
\newunicodechar{⸤}{\blhb}
\newunicodechar{⸥}{\brhb}

% Combine list of figures and tables. Generalised version of https://tex.stackexchange.com/a/14545/203081
\makeatletter
\providecommand\phantomsection{}% for hyperref
\newcommand{\listofillustrations}{%
	\chapter{\illustrationsname}%
	\phantomsection
	%\addcontentsline{toc}{chapter}{Lijst van illustraties}%
	\section*{\listfigurename}%
	\phantomsection
	\addcontentsline{toc}{section}{\listfigurename}%
	\@starttoc{lof}%
	\bigskip
	\section*{\listtablename}%
	\phantomsection
	\addcontentsline{toc}{section}{\listtablename}%
	\@starttoc{lot}%
	\bigskip
	%\section*{\listalgorithmname}%
	%\phantomsection
	\addcontentsline{toc}{section}{\listalgorithmname}%
	% So, for some reason, \@starttoc{loa} doesn't want to work (it outputs garbled titles). Luckily I found this fix https://tex.stackexchange.com/a/198860/203081 as an alternative. It lets \listofalgorithms do its thing as usual, except we ninja-edit \chapter* to be \section*.
	\begingroup
	\let\chapter\section% Temporary
	\listofalgorithms
	\endgroup
	%\@starttoc{loa}%
}
\makeatother

% Todonotes that doesn't kill you when you include it. TODO: Could be better, e.g. anti-overlap.
\let\marginnote\someundefinedcommand
\usepackage{marginnote}
%\reversemarginpar
\newcommand{\marginboxTB}[2]{\marginnote{%
	\begin{tikzpicture}%
		\node[draw,fill=#1,align=left,rounded corners,text width=0.5\marginparwidth] () {#2};%Need to align=... to be breakable. https://tex.stackexchange.com/a/31100/203081
	\end{tikzpicture}}[-1.75em]}% Move up by one line.

\newcommand{\todo}[1]{\marginboxTB{red!15}{#1}}

% --- Gradient-coloured table cells: surround cell content by \tgrad. Based on a mix of answers at https://tex.stackexchange.com/q/174998/203081.
\usepackage{pgf}
\usepackage{etoolbox}

% Definitions
\newcommand*{\MinNumber}{0.0}%
\newcommand*{\MidNumber}{0.5} %
\newcommand*{\MaxNumber}{1.0}%

\definecolor{high}{HTML}{03AC13}
\definecolor{mid}{HTML}{F7E379}
\definecolor{low}{HTML}{ec462e}
\newcommand*{\cellopacity}{80}

% Cell command
\newcommand{\tgrad}[1]{%
    \ifdim #1 pt > \MidNumber pt%
        \pgfmathparse{max(min(100.0*(#1 - \MidNumber)/(\MaxNumber-\MidNumber),100.0),0.00)}%
        \xdef\PercentColor{\pgfmathresult}%
        \cellcolor{high!\PercentColor!mid!\cellopacity}#1%
    \else
        \pgfmathparse{max(min(100.0*(\MidNumber - #1)/(\MidNumber-\MinNumber),100.0),0.00)}%
        \xdef\PercentColor{\pgfmathresult}%
        \cellcolor{low!\PercentColor!mid!\cellopacity}#1%
    \fi
}

% Optional: add a new tabular column type, "R", that applies tgrad to all cells automatically. Only works for fully numeric columns though; no column headers allowed.
%\usepackage{collcell}
%\newcolumntype{R}{>{\collectcell\tgrad}c<{\endcollectcell}}
